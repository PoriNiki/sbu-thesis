\cchapter{مقدمه}
در این بخش یک معرفی اجمالی از پایان نامه آورده می‌شود. تمام متنی که در این قالب آمده صرفاً جهت نمایش قالب و نشان دادن قابلیت‌های آن است.
این قالب با استفاده از بسته 
\lr{XePersian}
ساخته شده و از تمام قابلیت‌های رایج \lr{ LaTeX} می‌توان در نگارش آن استفاده کرد. و برای استفاده تمام دانشجویان دانشگاه شهید بهشتی منتشر شده است.
در بخش‌‌های بعدی به نحوه استفاده از این قالب می‌پردازیم. در ادامه متنی بی معنا جهت پر شدن حجم صفحه آورده می‌شود:

لورم ایپسوم ( به انگلیسی \lr{lorem ipsum} ) متنی بی مفهوم است که تشکیل شده از کلمات معنی دار یا بی معنی کنار هم. کاربر با دیدن متن لورم ایپسوم تصور میکند متنی که در صفحه مشاهده میکند این متن واقعی و مربوط به توضیحات صفحه مورد نظر است واقعی است. حالا سوال اینجاست که این متن « لورم ایپسوم » به چه دردی میخورد و اساسا برای چه منظور و هدفی ساخته شده است؟ پیش از بوجود آمدن لورم ایپسوم ، طراحان وب سایت در پروژه های وب سایت و طراحان کرافیک در پروژه های طراحی کاتولوگ ، بروشور ، پوستر و ... همواره با این مشکل مواجه بودند که صفحات پروژه خود را پیش از آنکه متن اصلی توسط کارفرما ارائه گردد و در صفحه مورد نظر قرار گیرد چگونه پر کنند؟؟ اکثر طراحان با نوشتن یک جمله مانند «این یک متن نمونه است» ویا «توضیحات در این بخش قرار خواهند گرفت» و کپی آن به تعداد زیاد یک یا چند پاراگراف متن میساختند که تمامی متن ها و کلمات ، جملات و پاراگراف ها تکراری بود و از این رو منظره خوبی برای بیننده نداشت و ضمنا به هیچ وجه واقعی به نظر نمیرسید تا بتواند شکل و شمایل تمام شده پروژه را نشان دهد. از این رو متنی ساخته شد که با دو کلمه ( به فارسی : لورم ایپسوم ) آغاز میشد وبا همین نام در بین طراحان وب و گرافیک شناخته و به سرعت محبوب شد. وب سایت های سازنده لورم ایپسوم میتوانند هر تعداد کلمه و پاراگراف که بخواهید به صوورت تکراری یا غیر تکراری برایتان بسازند و تحویلتان بدهند تا از آنها در پروژه هایتان استفاده کنید. ( لورم ایپسوم فارسی) متن های لورم ایپسوم را به زبان فارسی و علاوه بر زبان فارسی به انگلیسی ، عربی ، ترکی استانبولی و ... برایتان میسازد. زبان های دیگر نیز رفته رفته به بانک اطلاعاتی لورم ایسپوم فارسی اضافه خواهند شد.  لورم ایپسوم ( به انگلیسی \lr{lorem ipsum} ) متنی بی مفهوم است که تشکیل شده از کلمات معنی دار یا بی معنی کنار هم. کاربر با دیدن متن لورم ایپسوم تصور میکند متنی که در صفحه مشاهده میکند این متن واقعی و مربوط به توضیحات صفحه مورد نظر است واقعی است. حالا سوال اینجاست که این متن « لورم ایپسوم » به چه دردی میخورد و اساسا برای چه منظور و هدفی ساخته شده است؟ پیش از بوجود آمدن لورم ایپسوم ، طراحان وب سایت در پروژه های وب سایت و طراحان کرافیک در پروژه های طراحی کاتولوگ ، بروشور ، پوستر و ... همواره با این مشکل مواجه بودند که صفحات پروژه خود را پیش از آنکه متن اصلی توسط کارفرما ارائه گردد و در صفحه مورد نظر قرار گیرد چگونه پر کنند؟؟ اکثر طراحان با نوشتن یک جمله مانند «این یک متن نمونه است» ویا «توضیحات در این بخش قرار خواهند گرفت» و کپی آن به تعداد زیاد یک یا چند پاراگراف متن میساختند که تمامی متن ها و کلمات ، جملات و پاراگراف ها تکراری بود و از این رو منظره خوبی برای بیننده نداشت و ضمنا به هیچ وجه واقعی به نظر نمیرسید تا بتواند شکل و شمایل تمام شده پروژه را نشان دهد. از این رو متنی ساخته شد که با دو کلمه ( به فارسی : لورم ایپسوم ) آغاز میشد وبا همین نام در بین طراحان وب و گرافیک شناخته و به سرعت محبوب شد. وب سایت های سازنده لورم ایپسوم میتوانند هر تعداد کلمه و پاراگراف که بخواهید به صوورت تکراری یا غیر تکراری برایتان بسازند و تحویلتان بدهند تا از آنها در پروژه هایتان استفاده کنید. ( لورم ایپسوم فارسی) متن های لورم ایپسوم را به زبان فارسی و علاوه بر زبان فارسی به انگلیسی ، عربی ، ترکی استانبولی و ... برایتان میسازد. زبان های دیگر نیز رفته رفته به بانک اطلاعاتی لورم ایسپوم فارسی اضافه خواهند شد.  

\cchapter{معرفی قابلیت‌های قالب}
در این بخش با آوردن یک متن ساده به نمایش ظاهر و ساختار قالب و همچنین معرفی برخی دستورات لازم برای کار با آن قالب پرداخته می‌شود. این توضیحات بسیار مختصر بوده و صرفاً برای معرفی قالب می‌باشد و چنانچه با \lr{LaTeX} آشنایی ندارید، بهتر است پیش از آغاز تدوین پایان‌نامه مختصری در مورد نحوه کار با \lr{LaTeX} مطالعه بفرمایید.
\section{نگارش}
رعایت تمامی اصول نگارش در هنگام تدوین پایان‌نامه الزامیست، بسیاری از نکات نگارشی توسط قالب رعایت می‌شوند. در ادامه این بخش به معرفی برخی دستورات کاربردی برای این کار پرداخته می‌شود.
\\
به طور پیش‌‌فرض هر پاراگراف به صورت خودکار با فاصله  از کنار آغاز می‌شود.  چنانچه در حالت خاصی، نیاز به حذف این فاصله باشد می‌توانید از دستور 
\LTR{\lr{\textbackslash noindent}} \RTL{}\noindent
استفاده کنید.\\
برای نوشتن متون انگلیسی لازم است آن‌ها را داخل تگ
 \lr{\textbackslash lr\{\}}
 قرار دهید. به عنوان مثال با نوشتن 
 \lr{\textbackslash lr\{word\}}
 کلمه \lr{word} به درستی داخل متن قرار می‌گیرد.
 \par
 با نوشتن \textbackslash\textbackslash می‌توانید خط جدیدی را آغاز کنید و نوشتن \textbackslash par نیز باعث ایجاد یک پاراگراف جدید خواهد شد. با قرار دادن متن در داخل تگ \lr{\textbackslash textbf\{\}} متن به صورت \textbf{ضخیم} و با قرار دادن نوشته در داخل تگ \lr{\textbackslash textit\{\}} نوشته \textit{کج} خواهد شد. امکان استفاده هم‌زمان از این تگ‌ها نیز وجود دارد. برای آشنایی بیشتر با دستورات، پیشنهاد می‌شود به آموزش‌های \lr{LaTeX} مراجعه کنید.
 \par

 پانوشت‌‌ها یکی از بخش‌های اصلی در هر نوشته‌ای می‌باشد. در این قالب شماره‌های پانوشت در هر صفحه، مجدداً از ۱ آغاز می‌شود.  با نوشتن
\LTR{\lr{\textbackslash LTRfootnote\{footnote\}}} \RTL{} \noindent
  می‌توان یک پانوشت لاتین\LTRfootnote{footnote} اضافه نمود.
 \par
برای افزودن پانوشت‌های فارسی\RTLfootnote{یک پانوشت فارسی} نیز از دستور زیر استفاده می‌شود.
\LTR{\lr{\textbackslash RTLfootnote\{پانوشت\}}} \RTL{} \noindent
نمونه‌های افزودن پانوشت نیز در همین قسمت وجود دارد.






\section{بخش‌بندی}
برای ساخت یک فصل جدید کافیست از دستور
\LTR{\lr{\textbackslash cchapter\{عنوان\}}} \RTL{} \noindent
استفاده شود. با نوشتن این دستور به صورت خودکار یک فصل جدید اضافه شده و عنوان آن در یک صفحه مجزا قرار می‌گیرد.
هر بخش می‌تواند شامل تعدادی \lr{Section} باشد. شماره‌های آن مانند آنچه در بالا نیز مشاهده می‌کنید با . از یک‌دیگر جدا شده و به صورت خودکار شماره‌گزاری  شده و به فهرست اضافه می‌شوند. کافیست برای ساخت بخش دستور 
\LTR{\lr{\textbackslash section\{عنوان\}}} \RTL{}\noindent
را وارد کنید. همچنین با دستورات زیر می‌توانید زیربخش و حتی زیر زیر بخش، ایجاد کنید.
\LTR{\lr{\textbackslash subsection\{عنوان\}}} \RTL{}
\LTR{\lr{\textbackslash subsubsection\{عنوان\}}} \RTL{}\noindent
به عنوان مثال بخش زیر را در نظر بگیرید :(متون این زیربخش‌ها بی‌معنا و برای پر کردن صفحه می‌باشد.)

\subsection{یک زیر بخش}
اگر زیر بخش‌ها به سطح سوم برسند شماره‌گذاری نمی‌شوند ولی در فهرست مطالب قرار می‌گیرند، به عنوان مثال:
\subsubsection{زیر بخشی در زیر بخش}
متن
\subsubsection{ دومین زیربخش در زیربخش}
متنی دیگر

\subsection{زیر بخشی دیگر}
لورم ایپسوم ( به انگلیسی \lr{lorem ipsum} ) متنی بی مفهوم است که تشکیل شده از کلمات معنی دار یا بی معنی کنار هم. کاربر با دیدن متن لورم ایپسوم تصور میکند متنی که در صفحه مشاهده میکند این متن واقعی و مربوط به توضیحات صفحه مورد نظر است واقعی است. حالا سوال اینجاست که این متن « لورم ایپسوم » به چه دردی میخورد و اساسا برای چه منظور و هدفی ساخته شده است؟ پیش از بوجود آمدن لورم ایپسوم ، طراحان وب سایت در پروژه های وب سایت و طراحان کرافیک در پروژه های طراحی کاتولوگ ، بروشور ، پوستر و ... همواره با این مشکل مواجه بودند که صفحات پروژه خود را پیش از آنکه متن اصلی توسط کارفرما ارائه گردد و در صفحه مورد نظر قرار گیرد چگونه پر کنند؟؟ اکثر طراحان با نوشتن یک جمله مانند «این یک متن نمونه است» ویا «توضیحات در این بخش قرار خواهند گرفت» و کپی آن به تعداد زیاد یک یا چند پاراگراف متن میساختند که تمامی متن ها و کلمات ، جملات و پاراگراف ها تکراری بود و از این رو منظره خوبی برای بیننده نداشت و ضمنا به هیچ وجه واقعی به نظر نمیرسید تا بتواند شکل و شمایل تمام شده پروژه را نشان دهد. از این رو متنی ساخته شد که با دو کلمه ( به فارسی : لورم ایپسوم ) آغاز میشد وبا همین نام در بین طراحان وب و گرافیک شناخته و به سرعت محبوب شد. وب سایت های سازنده لورم ایپسوم میتوانند هر تعداد کلمه و پاراگراف که بخواهید به صوورت تکراری یا غیر تکراری برایتان بسازند و تحویلتان بدهند تا از آنها در پروژه هایتان استفاده کنید. ( لورم ایپسوم فارسی) متن های لورم ایپسوم را به زبان فارسی و علاوه بر زبان فارسی به انگلیسی ، عربی ، ترکی استانبولی و ... برایتان میسازد. زبان های دیگر نیز رفته رفته به بانک اطلاعاتی لورم ایسپوم فارسی اضافه خواهند شد.  لورم ایپسوم ( به انگلیسی \lr{lorem ipsum} ) متنی بی مفهوم است که تشکیل شده از کلمات معنی دار یا بی معنی کنار هم. کاربر با دیدن متن لورم ایپسوم تصور میکند متنی که در صفحه مشاهده میکند این متن واقعی و مربوط به توضیحات صفحه مورد نظر است واقعی است. حالا سوال اینجاست که این متن « لورم ایپسوم » به چه دردی میخورد و اساسا برای چه منظور و هدفی ساخته شده است؟ پیش از بوجود آمدن لورم ایپسوم ، طراحان وب سایت در پروژه های وب سایت و طراحان کرافیک در پروژه های طراحی کاتولوگ ، بروشور ، پوستر و ... همواره با این مشکل مواجه بودند که صفحات پروژه خود را پیش از آنکه متن اصلی توسط کارفرما ارائه گردد و در صفحه مورد نظر قرار گیرد چگونه پر کنند؟؟ اکثر طراحان با نوشتن یک جمله مانند «این یک متن نمونه است» ویا «توضیحات در این بخش قرار خواهند گرفت» و کپی آن به تعداد زیاد یک یا چند پاراگراف متن میساختند که تمامی متن ها و کلمات ، جملات و پاراگراف ها تکراری بود و از این رو منظره خوبی برای بیننده نداشت و ضمنا به هیچ وجه واقعی به نظر نمیرسید تا بتواند شکل و شمایل تمام شده پروژه را نشان دهد. از این رو متنی ساخته شد که با دو کلمه ( به فارسی : لورم ایپسوم ) آغاز میشد وبا همین نام در بین طراحان وب و گرافیک شناخته و به سرعت محبوب شد. وب سایت های سازنده لورم ایپسوم میتوانند هر تعداد کلمه و پاراگراف که بخواهید به صوورت تکراری یا غیر تکراری برایتان بسازند و تحویلتان بدهند تا از آنها در پروژه هایتان استفاده کنید. ( لورم ایپسوم فارسی) متن های لورم ایپسوم را به زبان فارسی و علاوه بر زبان فارسی به انگلیسی ، عربی ، ترکی استانبولی و ... برایتان میسازد. زبان های دیگر نیز رفته رفته به بانک اطلاعاتی لورم ایسپوم فارسی اضافه خواهند شد.  

\section{ارجاعات}